% Options for packages loaded elsewhere
\PassOptionsToPackage{unicode}{hyperref}
\PassOptionsToPackage{hyphens}{url}
%
\documentclass[
  openany]{book}
\usepackage{amsmath,amssymb}
\usepackage{lmodern}
\usepackage{iftex}
\ifPDFTeX
  \usepackage[T1]{fontenc}
  \usepackage[utf8]{inputenc}
  \usepackage{textcomp} % provide euro and other symbols
\else % if luatex or xetex
  \usepackage{unicode-math}
  \defaultfontfeatures{Scale=MatchLowercase}
  \defaultfontfeatures[\rmfamily]{Ligatures=TeX,Scale=1}
\fi
% Use upquote if available, for straight quotes in verbatim environments
\IfFileExists{upquote.sty}{\usepackage{upquote}}{}
\IfFileExists{microtype.sty}{% use microtype if available
  \usepackage[]{microtype}
  \UseMicrotypeSet[protrusion]{basicmath} % disable protrusion for tt fonts
}{}
\makeatletter
\@ifundefined{KOMAClassName}{% if non-KOMA class
  \IfFileExists{parskip.sty}{%
    \usepackage{parskip}
  }{% else
    \setlength{\parindent}{0pt}
    \setlength{\parskip}{6pt plus 2pt minus 1pt}}
}{% if KOMA class
  \KOMAoptions{parskip=half}}
\makeatother
\usepackage{xcolor}
\usepackage{longtable,booktabs,array}
\usepackage{calc} % for calculating minipage widths
% Correct order of tables after \paragraph or \subparagraph
\usepackage{etoolbox}
\makeatletter
\patchcmd\longtable{\par}{\if@noskipsec\mbox{}\fi\par}{}{}
\makeatother
% Allow footnotes in longtable head/foot
\IfFileExists{footnotehyper.sty}{\usepackage{footnotehyper}}{\usepackage{footnote}}
\makesavenoteenv{longtable}
\usepackage{graphicx}
\makeatletter
\def\maxwidth{\ifdim\Gin@nat@width>\linewidth\linewidth\else\Gin@nat@width\fi}
\def\maxheight{\ifdim\Gin@nat@height>\textheight\textheight\else\Gin@nat@height\fi}
\makeatother
% Scale images if necessary, so that they will not overflow the page
% margins by default, and it is still possible to overwrite the defaults
% using explicit options in \includegraphics[width, height, ...]{}
\setkeys{Gin}{width=\maxwidth,height=\maxheight,keepaspectratio}
% Set default figure placement to htbp
\makeatletter
\def\fps@figure{htbp}
\makeatother
\setlength{\emergencystretch}{3em} % prevent overfull lines
\providecommand{\tightlist}{%
  \setlength{\itemsep}{0pt}\setlength{\parskip}{0pt}}
\setcounter{secnumdepth}{5}
\ifLuaTeX
  \usepackage{selnolig}  % disable illegal ligatures
\fi
\IfFileExists{bookmark.sty}{\usepackage{bookmark}}{\usepackage{hyperref}}
\IfFileExists{xurl.sty}{\usepackage{xurl}}{} % add URL line breaks if available
\urlstyle{same} % disable monospaced font for URLs
\hypersetup{
  pdftitle={A story of 3 dogs},
  pdfauthor={Heather Pretty},
  hidelinks,
  pdfcreator={LaTeX via pandoc}}

\title{A story of 3 dogs}
\author{Heather Pretty}
\date{}

\begin{document}
\maketitle

{
\setcounter{tocdepth}{1}
\tableofcontents
}
\hypertarget{introduction}{%
\chapter{Introduction}\label{introduction}}

A short book about the 3 dogs I've had in my life.

\hypertarget{tinker}{%
\chapter{Tinker}\label{tinker}}

Tinker was an Irish Setter.

He was red and big and loved to chase sticks and his red ball.
Because he was an Irish Setter, his nature was to be a bit crazy and goofy.

He showed the girl what it was to have love from a dog.
When she was young, she would place her head between his front paws when he was laying down, and would feel oh so safe.

When Tinker died, the girl found a place to be alone to mourn his loss.
The girl vowed she would someday have another dog to show her children what it is to have love from a dog.

\hypertarget{sophie}{%
\chapter{Sophie}\label{sophie}}

Sophie was a Black Labrador Retriever.

She loved to fetch balls and play ``goalie'' between the kitchen doors.
Because she was a Labrador Retriever, her nature was to be loving and protecting.

Sophie was there for the woman when she moved to the new country.
Sophie was a constant companion, and never let the woman feel alone even when the new country felt strange.
When the woman had children, Sophie loved them and was patient and gentle - even when the children pulled on her tail.
Sophie taught the children what it means to be loved by a dog.

The woman loved Sophie when she was a puppy with fur as black as coal.
The woman loved Sophie more and more each year - even as she saw the sprinkles of gray grow around her nose and face.
The woman learned from Sophie that old dogs should get whatever they want.

When Sophie died, the woman and her husband buried her under a dogwood tree.

\hypertarget{dewey}{%
\chapter{Dewey}\label{dewey}}

Dewey is an English Setter.

He loves to chase balls and toys and then play ``keep away''.
When he gets excited and happy he bounces on his front legs.

Because Dewey is an English Setter, his nature is to be OCD and quite crazy.

When the woman's daughter was having trouble with pandemic lock downs and missing the dog Sophie,
Dewey magically appeared into the life of the family and brought joy - and very sharp puppy teeth.

Dewey teaches the woman to be patient, and that love is more important than having unchewed rugs.
Dewey teaches the woman's husband to be patient, and that no matter how many patches made, there is always another hole to find under the fence.
Dewey teaches the family that when you need affection, you should speak up and bark until someone pulls you into their arms.
Dewey teaches the family that snuggles are so important, and the best way to hug is to lean in with your whole heart.

\end{document}
